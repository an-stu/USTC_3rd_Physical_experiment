\documentclass[utf8]{ctexart}


\usepackage{amssymb}
\usepackage{enumerate}
\usepackage[numbers]{natbib} 
\usepackage{geometry}
\geometry{left=3.0cm,right=3.0cm,top=2.5cm,bottom=2.5cm}
\usepackage{fancyhdr}
\pagestyle{fancy}
\setlength{\headheight}{10mm}
\rhead{}
\lhead{}
\fancyhead[C]{
	\begingroup
	\setlength{\tabcolsep}{10pt} % Default value: 6pt
	\renewcommand{\arraystretch}{1.5} % Default value: 1
	\begin{tabular}{ccc}
		& \large{\textbf{氢氘光谱实验报告}} &  \\
		少年班学院 \qquad \qquad & 刘子安 PB20000069 & \qquad \today
	\end{tabular}
	\endgroup
}   %在此处插入作者信息,改变页眉,此页眉是由我设计的,类似于实验报告纸
\fancyfoot[C]{ 第 {\thepage} 页,共 \pageref{unknown} 页}
\renewcommand{\headrulewidth}{0pt}


\usepackage{graphicx}

\usepackage{siunitx} % 单位

\begin{document}

	\subsection*{1.实验目的}	
    \begin{itemize}
        \item [1.] 了解氢氘光谱实验仪的基本构造和使用方法,知道如何使用光栅光谱仪
        测量光谱。
        \item [2.] 知道同位素原子光谱的区别,学会测量氢氘巴耳末系的前四条谱线。
        \item [3.] 可以通过氢氘巴耳末系的前四条谱线计算里德伯常数、氢氘质量比以及质子电子质量比。
    \end{itemize}
	\subsection*{2.实验原理} 
	具有相同质子数,不同中子数同一元素的不同核素互为同位素。
    在谱线上,同位素对应的谱线会发生移位,称为同位素移位。

    由原子物理的知识可以知道,氢原子巴耳末系的谱线波长满足规律:
    \begin{equation} \label{lambda}
        \frac{1}{\lambda} = R\left(\frac{1}{2^2}-\frac{1}{n^2}\right) = \frac{R_{\infty}}{1+m_e/M}\left(\frac{1}{2^2}-\frac{1}{n^2}\right)
    \end{equation}
    其中
    \begin{equation}
        R_{\infty} = \frac{2\pi^2me^4}{(4\pi\varepsilon_0)^2ch^3}
    \end{equation}
    $h$是普朗克常数,$c$是光速,$\varepsilon_0$为真空中介电常数。

    氘是氢的同位素,它们有相同的质子和核外电子,只是氘比氢多了一个中子而使原子核的质量
    发生变化,从而使它的里德伯常量值也发生变化,两者的里德伯常量分别为:
    \begin{equation} \label{RH}
        R_H = \frac{R_\infty}{1+m_e/M_H}
    \end{equation}
    \begin{equation} \label{RD}
        R_D = \frac{R_\infty}{1+m_e/M_D}
    \end{equation}
    其中$M_H$和$M_D$分别表示氢与氘原子核的质量,由(\ref{RH})和(\ref{RD})式以及
    两同位素的巴尔末公式(\ref{lambda})将可解得:
    \begin{equation}
        \frac{M_D}{M_H} = \frac{m}{M_H}\times\frac{\lambda_H}{(\lambda_D-\lambda_H+\lambda_D\frac{m}{M_H})}
    \end{equation}
    利用测量的波长计算氢氘谱峰的波长差:
    \begin{equation}
            \Delta\lambda = \lambda_H-\lambda_D = \left( \frac{1}{R_H} - \frac{1}{R_D} \right)
            \bigg/\left( \frac{1}{2^2} - \frac{1}{n^2} \right) \approx 
            \frac{\frac{M+m}{M}-\frac{2M+m}{2M}}{1/\lambda}=\frac{m}{2M}\lambda
    \end{equation}
    即有:
    \begin{equation}
        \frac{M}{m} = \frac{\lambda}{2\Delta\lambda}
    \end{equation}

	\subsection*{3.实验仪器}
	电源、光电倍增管、光谱仪、汞灯、氢氘灯。
    \vspace{4em}
	
	\subsection*{4.实验数据处理}
	\subsubsection*{(1).实验数据表格}
    对汞灯谱线的测量数据如下:
	\begin{table}[htbp]
        \centering
        \caption{汞灯测量数据}
        \begin{tabular}{|c|c|c|c|c|c|}
            \hline
            汞灯谱线序号 & 1 & 2 & 3 & 4 & 5 \\
            \hline
            波长$/\unit{nm}$ & 365.02 & 365.49 & 366.30 & 404.61 & 407.72 \\
            \hline
            汞灯谱线序号 & 6 & 7 & 8 & 9 & \\
            \hline
            波长$/\unit{nm}$ & 435.85 & 546.04 & 576.84 & 578.98 & \\
            \hline
            
        \end{tabular}
    \end{table}

    氢氘灯的测量数据:

    \begin{table}[htbp]
        \centering
        \caption{氢氘灯测量数据}
        \begin{tabular}{|c|c|c|c|c|}
            \hline
            能级n & 3 & 4 & 5 & 6 \\
            \hline
            波长 $\lambda_H/\unit{nm}$ & 656.50 & 486.08 & 434.02 & \\
            \hline
            波长 $\lambda_D/\unit{nm}$ & 656.70 & 485.94 & 433.87 & \\
            \hline
        \end{tabular}
    \end{table}
	由于时间问题,未能调整好能级为6时对应的谱线,所以没有测出该能级对应的波长。
	\subsubsection*{(2).误差分析}
	
	\subsubsection*{(3).实验讨论}
	
	\subsection*{5.思考题}
	\begin{itemize}
		\item [1.]
		\item [ ]答:
		\item [2.]
		\item [ ]答:
		\item [3.]
		\item [ ]答:
		\item [4.]
		\item [ ]答:
	\end{itemize}
	
	
	\subsection*{6.总结}
	
	\section*{致谢}
	
	\begin{center}
		感谢\cite{bk1}大学物理实验中心和韦先涛老师
	\end{center}
	\bibliographystyle{plain}
	\bibliography{example} % 同文件夹下新建所需的example.bib文件
	
	
	\label{unknown}
\end{document}